% Turing Machine Definition and TiKz depiction.


% Head for reading/writing
% Alphabet: blank, 0,1
% Tape Alphabet: Alphabet with a schwa i.e. schwa can't be printed by the head.
% m-configurations i.e. states:
%
%               q_i S_j PS_k,L q_m
%
%     To be read as: if in state q_i and reading symbol S_j, then print symbol S_k and move the head left. Machine to state q_m.

% Aside notes: Alphabet can have more characters e.g. "characters for working" as in Turing's original paper. Also the actual printed characters can be expanded too. 

% Data representation is another point, apart from the issue of defining the machine. We can use unary, or binary presentations of numbers. What ever suits the particular sequence/number/function we want to compute.
