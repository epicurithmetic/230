\documentclass{article}

% For each new section, use the following for a heading.
% \vspace{2em}
% {\bf Section Heading}

\begin{document}

\begin{center}
{\bf Introduction Lecture: Out of the Mire}
\end{center}
\hrule
\vspace{1em}

% Provide a short explanation of the aim for today's lecture. Where does it fit in the narrative of the course?
In this lecture we will think about how mathematics has changed over the millenia that humans have developed it. How the nature of the objects that mathematicians think about have become more complex, further withdrawn from the physical world of experience, and as such more difficult to critically evaluate.


\vspace{2em}
{\bf Theorems and Proofs}

All statements of fact must be argued for. Evidence must be collected and presented in such a way as to prove that the statement is true. Mathematics (in the sense that it is a collection of knowledge) is precisely a collection of statements with the associated proofs: this collection is made of papers and books with theorems (the statements of fact) and their proofs.

If we pick anyone of these papers, then we will see that the statements (and proofs) are written in terms which themselves have complex definitions and their own proofs. Thus to properly evaluate any particular statement, we must work our way back through the references to check all of the foundational statements.

\begin{center}
Question: When does this process through the past papers stop?
\end{center}

Even if we reach the end of the line, then this ``first paper" will make statements in some language, providing proof using some terms. But these final terms and statments will them selves need proof!


\vspace{2em}
{\bf M{\"u}nchausen's Trilemma}

It's not at all clear how we stop this seemingly infinite regress. If we are honest, I believe we can't really get outselves out of this: this trilemma seems to point to some epistemological limit, at least for our way of thinking. This problem is not unique to mathematics, many other thinkers from other disciplines have come across it. There are a number of philosophical positions that have been taken to ``get out" of the this regress. Often the idea is presented using the following options:

\begin{itemize}
  \item[] (Circular Argument) Use the claimed fact in the proof.
  \item[] (Regressive Argument) Continue proving statements ad infinitum.
  \item[] (Axioms) Decide on some obvious truths and write all proofs from these.
\end{itemize}

Posing the problem in this way leaves us in a trilemma. Often called M{\"u}nchausen's Trilemma in reference to a story where the fictional Baron M{\"u}nchausen uses his hair to pull himself and his mule out of a mire. If we hope for some sense of absolute undeniable truth, then none of these options seem particularly statisfying. Which ever path we take, we will essentially be pulling ourselves up by our bootstraps. This is no easy task.

\vspace{2em}
{\bf Foundationalism in Mathematics}

\newpage
\hrule
\vspace{1em}
{\bf Summary:}
% Provide a short summary of the key points to take away from the lecture. Relate back to the introduction statements and the narrative of the course.

{\bf Further Reading:}
% Provide a short list of some references to learn more, or gain another perspective, about the key learning points.
\begin{itemize}
  \item https://en.wikipedia.org/wiki/M%C3%BCnchhausen_trilemma

\end{itemize}
\end{document}
