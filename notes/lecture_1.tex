\documentclass{article}

% Import the style file
\usepackage{230style}

% Bibliography
\usepackage[]{biblatex}   % First argument is for the bibliography style.
\bibliography{230Bib.bib}


\begin{document}

\section*{Lecture 1: Out of the Mire}

\epigraph{Every demonstrable science must start with indemonstrable principles. Otherwise the steps of demonstration would be endless}{Aristotle}

\subsection*{\indent The Mire}

% In this section we will consider what it takes to prove a statement in mathematics. Leading the reader to the Munchausen Trilemma.

\subsection*{\indent Euclid's Elements}

% This Trilemma was known to the Ancient Greeks. Euclid wrote his book The Elements in what we would now call the "foundationalist" response to the Trilemma.

\subsection*{\indent Foundational Crisis}

% Explain that the richness of mathematics requires new foundations.

\subsection*{\indent Hilbert's Program}

% Concluding remarks outlining the topics of the course and their place in this historial context.

\newpage

Further content from the first lecture.

\begin{defn}[Turing Machine]

This is a Turing Machine

\end{defn}

Thanks to \cite{jdhlectures} for writing an interesting book.





\printbibliography
\end{document}
