% Document for practicing with the Gentzen style natural deductions in LaTeX
% http://logicmatters.net/resources/ndexamples/proofsty.html

\documentclass{article}

% Import the style file
%\usepackage{230style}
\usepackage{proof}
\usepackage{cancel}
\begin{document}


% a implies a
\infer[\rightarrow I,1]
{\alpha \rightarrow \alpha}
{\infer[1]
{\alpha}
{}}

\vspace{5em}

% Modus ponens
\infer[\rightarrow E]
{\beta}
{\alpha & \alpha \rightarrow \beta}

\vspace{5em}

\infer[\rightarrow E]{\gamma}{\infer[\rightarrow E]{\lnot \beta \rightarrow \gamma}{\alpha & \alpha \rightarrow (\lnot \beta \rightarrow \gamma)} & \infer[\rightarrow E]{\lnot \beta}{\alpha & \alpha \rightarrow \lnot \beta}}

\vspace{5em}

Gentzen style proof of {\bf Modus Tollens} $\{\lnot \beta, \alpha\rightarrow \beta \} \vdash \lnot \alpha$

\infer[\rightarrow I ]{\alpha \rightarrow \bot}
{\infer[\rightarrow E]{\bot}{\infer[]{\beta}{\infer[]{\cancel\alpha}{} & \alpha\rightarrow \beta} & \lnot \beta}}

\vspace{5em}
\newpage
Gentzen style proof of {\bf Law of Contraposition} with Modus Tollens

\infer[\to I]{\lnot\beta \rightarrow \lnot\alpha}
{\infer[\textnormal{MT}]{\lnot\alpha}
{\alpha\rightarrow\beta & \infer[]{\cancel\lnot\beta}{}}}

\vspace{5em}

Gentzen style proof of {\bf Law of Contraposition} $\alpha \rightarrow \beta \vdash \lnot\beta \rightarrow \lnot\alpha$

\infer[\to I]{\lnot\beta \rightarrow \lnot \alpha}
{\infer[\to I]{\alpha \rightarrow \bot}
{\infer[\to E]{\bot}
{\infer[\to I]{\beta}{\alpha \rightarrow \beta & \infer[]{\cancel\alpha}{}} & \infer[]{\cancel\lnot\beta}{}}}}

\newpage

Gentzen style proof of the {\bf Commutativity of Disjunction} $(\alpha \lor \beta) \vdash (\beta \lor \alpha)$

\infer[]{\beta \lor \alpha}
{\alpha \lor \beta & \infer[\to I]{\alpha\rightarrow(\beta \lor \alpha)}{\infer[\lor I]{\beta \lor \alpha}{\infer[]{\alpha}{}}} & \infer[\to I]{\beta\rightarrow(\beta \lor \alpha)}{\infer[\lor I]{\beta \lor \alpha}{\infer[]{\beta}{}}}}

\newpage

Gentzen style proof of the {\bf Associativity of Disjunction}

\begin{center}
$(\alpha \lor \beta) \lor \gamma \vdash \alpha \lor (\beta \lor \gamma)$
\end{center}


\infer[\lor E]{\alpha \lor (\beta \lor \gamma)}
{(\alpha \lor \beta) \lor \gamma
& \infer[\to I]{\gamma \rightarrow (\alpha \lor (\beta \lor \gamma))}
{\infer[\lor I]{\alpha \lor (\beta \lor \gamma)}
{\infer[\lor I]{\beta \lor \gamma}
{\infer[]{\cancel\gamma}
{}}}}
& \infer[\to I]{(\alpha \lor \beta) \rightarrow (\alpha \lor (\beta \lor \gamma))}
{\infer[\lor E]{\alpha \lor (\beta \lor \gamma)}
{\infer[]{\cancel{\alpha \lor \beta}}{}
& \infer[\to I]{\alpha \rightarrow (\alpha \lor (\beta \lor \gamma))}
{\infer[\lor I]{\alpha \lor (\beta \lor \gamma)}
{\infer[]{\cancel\alpha}
{}}}
&
\infer[\to I]{\beta\rightarrow((\alpha \lor (\beta \lor \gamma))}
{\infer[\lor I]{\alpha \lor (\beta \lor \gamma)}
{\infer[\lor I]{\beta \lor \gamma}
{\infer[]{\cancel\beta}
{}}}}}}}


\end{document}
